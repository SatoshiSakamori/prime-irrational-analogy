\documentclass[11pt]{article}
\usepackage{amsmath, amssymb, amsthm}
\usepackage{geometry}
\usepackage{hyperref}
\geometry{a4paper, margin=25mm}
\title{Analogy Between Prime Sieve and Irrational Sieve}
\author{Satoshi Sakamori}
\date{\today}

\begin{document}
\maketitle

\begin{abstract}
This paper explores a structural analogy between the sieve of Eratosthenes, 
used to isolate prime numbers, and an analogous operation on the real numbers 
that isolates irrational numbers by successively removing rationals 
with increasing denominators. 
Both can be interpreted as infinite filtering processes that remove 
finite or periodic structures, leaving behind elements that cannot 
be finitely represented.
\end{abstract}

\section{Introduction}

The sieve of Eratosthenes removes numbers that are multiples of smaller integers, 
leaving behind primes---numbers that cannot be expressed as products of 
smaller integers. 

This paper proposes an analogous process on the real number line:
starting with the continuum $\mathbb{R}$, 
we remove all rational numbers that can be expressed with denominators 
below a given bound, and let that bound tend to infinity. 
The remaining set, in the limit, consists of the irrational numbers---those 
that cannot be finitely expressed as ratios of integers.

\section{The Sieve of Eratosthenes as a Filtering Process}

Let $\mathbb{N} = \{2, 3, 4, \ldots\}$. 
For each integer $p \ge 2$, define the set of multiples:
\[
M_p = \{ n \in \mathbb{N} \mid p \text{ divides } n \}.
\]
The sieve of Eratosthenes removes all such $M_p$ for each $p$ in increasing order, 
starting from $p=2$.

Formally, define the remaining set after sieving up to $p = P$ as
\[
S_P = \mathbb{N} \setminus \bigcup_{p \le P} M_p.
\]
Then, the primes are obtained in the limit:
\[
\mathbb{P} = \lim_{P \to \infty} S_P.
\]
This process removes all integers with finite factorizations involving 
smaller primes, leaving only those which cannot be decomposed---the primes themselves.

\section{An Analogous Sieve on the Real Numbers}

Now, consider the real line $\mathbb{R}$. 
For each positive integer $q$, define the set of rationals with denominator $q$:
\[
R_q = \left\{ \frac{p}{q} \in \mathbb{Q} \mid p \in \mathbb{Z} \right\}.
\]
These sets form a countable union:
\[
\mathbb{Q} = \bigcup_{q=1}^{\infty} R_q.
\]

We define a filtration of $\mathbb{R}$ analogous to the prime sieve:
\[
T_Q = \mathbb{R} \setminus \bigcup_{q \le Q} R_q.
\]
Then, as $Q \to \infty$, the limit set is
\[
\lim_{Q \to \infty} T_Q = \mathbb{R} \setminus \mathbb{Q},
\]
which is precisely the set of irrational numbers.

This ``irrational sieve'' removes all points that can be expressed as ratios 
of finite integers, just as the Eratosthenes sieve removes all numbers 
that can be expressed as products of smaller integers.

\section{Structural Analogy}

Both sieves share the following structural properties:
\begin{enumerate}
    \item \textbf{Base domain:} The natural numbers $\mathbb{N}$ vs. 
          the real numbers $\mathbb{R}$.
    \item \textbf{Finite representability:} 
          Integers with finite factorizations (composites) correspond 
          to real numbers with finite rational representations.
    \item \textbf{Filtering rule:} 
          Removal of elements generated by finite combinations 
          (products or rational divisions).
    \item \textbf{Limit structure:} 
          The remaining sets (primes and irrationals) 
          are those that cannot be expressed by any finite process 
          within the generating rule.
\end{enumerate}

Thus, primes and irrationals both emerge as residual entities 
beyond finite generative closure:
\[
\text{Primes: } \mathbb{P} = \mathbb{N} \setminus 
\bigcup_{p} M_p,
\quad
\text{Irrationals: } \mathbb{R} \setminus \bigcup_{q} R_q.
\]

\section{Conclusion and Outlook}

Both sieves can be seen as instances of a broader concept: 
\textit{a filtration process that removes all finitely generable elements 
to reveal the infinitary residue.}
In this sense, primes and irrationals occupy analogous positions 
in their respective domains.

This analogy opens potential investigations into:
\begin{itemize}
    \item Information-theoretic measures of ``finite generability''.
    \item Topological or measure-theoretic parallels between 
          $\mathbb{P}$ and $\mathbb{R}\setminus\mathbb{Q}$.
    \item Computational analogs of ``sieve complexity'' 
          across discrete and continuous systems.
\end{itemize}

\section*{Acknowledgement}

The conceptual inspiration for this analogy originates from 
a thought experiment by Satoshi Sakamori, 
who proposed viewing irrational numbers as the residue left after 
successively removing all rational partitions from the real line, 
in analogy with the sieve of Eratosthenes.

\end{document}