\documentclass[12pt]{article}
\usepackage{amsmath,amssymb,amsthm}
\usepackage{geometry}
\usepackage{setspace}
\usepackage{hyperref}

\geometry{a4paper, margin=1in}
\setstretch{1.3}

\title{On a Conceptual Analogy Between the Distribution of Prime Numbers and the Structure of Irrational Numbers}
\author{Satoshi Sakamori (draft)}
\date{\today}

\begin{document}
\maketitle

\begin{abstract}
This note explores a conceptual analogy between two seemingly unrelated sets: prime numbers and irrational numbers. 
Both can be viewed as \emph{residuals} that remain after systematically removing regular, periodic, or rationally describable structures from the number line. 
We discuss a heuristic and structural comparison, and propose a formal framework to examine this analogy.
\end{abstract}

\section{Motivation}

Prime numbers are the fundamental building blocks of integers, defined as numbers greater than one that cannot be expressed as the product of two smaller positive integers.  
Irrational numbers, on the other hand, are real numbers that cannot be expressed as the ratio of two integers.

The intuitive observation motivating this note is that both sets---primes and irrationals---can be understood as what remains when certain regular or periodic subsets are systematically eliminated from the number line.

\section{Elimination Process as a Structural Analogy}

\subsection{Prime Numbers as Residuals of Integer Sieve}

Consider the classical sieve of Eratosthenes.  
We begin with the set of integers $\mathbb{N} = \{2, 3, 4, \dots\}$ and remove all multiples of $2$, then of $3$, then of $5$, and so on.  
At each step, we eliminate elements of the form
\[
n = k p_i, \quad k \in \mathbb{N},
\]
where $p_i$ is a prime number identified in a previous step.  
The remaining integers---which cannot be expressed as multiples of smaller factors---are precisely the prime numbers.

In this view, primes are the ``non-periodic residues'' left after removing all regularly spaced (periodic) subsets.

\subsection{Irrational Numbers as Residuals of Rational Fractions}

Now consider an analogous procedure in $\mathbb{R}$:  
start with the interval $[0,1]$ and successively remove all rational numbers that can be expressed as $\frac{p}{q}$ for integers $p,q$.  
Formally, at step $q$, remove all numbers of the form
\[
x = \frac{p}{q}, \quad p \in \mathbb{Z}.
\]
After infinitely many steps, the remaining set is precisely the set of irrational numbers in $[0,1]$.

Again, these are the numbers that cannot be represented by regular, rational patterns.  
They occupy the ``gaps'' left after all rational structures are removed.

\section{Analogy and Conjectural Structure}

Both processes---the sieve for primes and the removal of rationals---can be seen as iterative elimination of structured or periodic subsets:
\begin{itemize}
    \item In $\mathbb{N}$, periodicity arises from modular arithmetic: multiples of $p$ form arithmetic progressions with period $p$.
    \item In $\mathbb{R}$, rational numbers $\frac{p}{q}$ form a dense but countable set that corresponds to ``rational partitions'' of the unit interval.
\end{itemize}

In both cases, the remainder is ``non-periodic'':
\begin{itemize}
    \item Primes: non-multiplicative residues under integer factorization.
    \item Irrationals: non-rational residues under rational partitioning.
\end{itemize}

\paragraph{Heuristic conjecture.}  
If we consider periodicity as a manifestation of rational structure, then primes and irrationals could be seen as distinct manifestations of ``aperiodic completeness'' in different numerical domains.  
This analogy invites a formal exploration of whether measures of irregularity or entropy applied to the distribution of primes and to continued fraction expansions of irrationals exhibit structural parallels.

\section{Future Work}

Potential directions for formalization include:
\begin{itemize}
    \item Define an entropy-like measure for the ``irregularity'' of the prime indicator function.
    \item Compare it to entropy of continued fraction coefficients of typical irrationals.
    \item Explore whether a shared measure of ``rational deficiency'' exists between $\mathbb{N}$ and $\mathbb{R}$.
\end{itemize}

\section{Remarks}

This note is purely heuristic and intended to provoke conceptual exploration.  
No claims of mathematical equivalence are made.  
The analogy may serve as a starting point for formal investigations into shared structures underlying multiplicative and rational aperiodicity.

\vspace{1em}
\noindent\textbf{Keywords:} prime numbers, irrational numbers, sieve, rationality, aperiodicity, entropy.

\end{document}